\documentclass{article}

\begin{document}

\paragraph{(1a)}
The height of the recursion tree is equal to the length of the path from the root to the tree. $T(n)=4T(n-1)$ and $T(0)=1$, so $T(n)=4^{n}$. The tree will be perfectly balanced, with each node with children having 4 children, so $h=log_4(T(n))$, which means $h=4n$.

\paragraph{(1b)}
Let $n(i)$, be the nodes at level i. $n(0) = 3$. Each node has 4 children, so $n(i+1)=4n(i)$. Consequently, $n(i)=3\cdot4^i$.

\paragraph{(1c)}
A node represents one call to snowflake edge, which corresponds to one triangle, so $\Theta(1)$.

\paragraph{(1d)}
Triangle count at level i = Number of nodes at i * Triangle count per node = $3\cdot4^i\cdot\Theta(1)=4^i$.

\paragraph{(1e)}
The asymptotic cost for the CPU is equal to the total number of triangles. $cost = n(n) = 4^n$.

\paragraph{(1f)}
$T(0)=3, T(n+1)=4T(n).$ For level i, if we have T(i) line segments, for each line segment, we will want 3 more points for the triangle, so $T(i+1)=T(i)+3T(i)=4T(i)$. From similar analysis to 1a, height=4n.

\paragraph{(1g)}
4

\paragraph{(1h)}
1 node = 1 line segment. $\Theta(1)$.

\paragraph{(1i)}
$\Theta(1)$.

\paragraph{(1j)}
$\Theta(4^n)$.

\paragraph{(1k)}
$\Theta(4^n)$.

\paragraph{(1l)}
$\Theta(4^n)$.

\paragraph{(1m)}
$\Theta(4n)$. Rasterization doesn't affect the height of the tree.

\paragraph{(1n)}
$3\cdot4^i$.

\paragraph{(1o)}
0, because you don't render if you're not at the end of the tree. 

\paragraph{(1p)}
$\Theta(\frac{1}{3}^n)$.

\paragraph{(1q)}
0. The algorithm doesn't render in the middle of recursion.

\paragraph{(1r)}
$3\cdot4^n\cdot\Theta(\frac{1}{3}^n) = \Theta(\frac{4}{3}^i)$.

\paragraph{(1s)}
The total cost is the cost of the leaves, which is $\Theta(\frac{4}{3}^i)$.

\paragraph{(1t)}
For a node at the root, the area is 1. For the next one, the total area is increased by the area of three triangles, each with side length 1/3 of the original. The area is $\Theta(l^2)$, so it costs 1/9 to render each one, so 1/3 additional for the next level. This means $T(n+1)=4/3T(n)$, which implies $T(n)=\Theta(\frac{4}{3}^n)$.

\paragraph{(1u)}

\paragraph{(2a)}
_find_min takes the longest.

\paragraph{(2b)}
_find_min is called 259964 times.

\paragraph{(2c)}
$O(n)$.

\paragraph{(2d)}
$O(logn)$.


\end{document}
